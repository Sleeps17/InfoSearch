\documentclass[12pt,a4paper]{article}

\usepackage[a4paper,margin=2cm]{geometry}
\usepackage{iftex}
\usepackage{graphicx}

\ifPDFTeX
  \usepackage[T2A]{fontenc}
  \usepackage[utf8]{inputenc}
  \usepackage[russian]{babel}
  \usepackage{lmodern}
\else
  \usepackage{fontspec}
  \usepackage[russian]{babel}
  \defaultfontfeatures{Ligatures=TeX}
  \setmainfont{Times New Roman}
\fi

\usepackage{microtype}
\usepackage{setspace}
\setstretch{1.2}

\usepackage{hyperref}
\hypersetup{
  colorlinks=true,
  linkcolor=blue,
  urlcolor=blue
}

\usepackage{booktabs}
\usepackage{tabularx}

\usepackage{enumitem}
\setlist[itemize]{leftmargin=1.5em, itemsep=0.15em, topsep=0.25em}

\usepackage{listings}
\usepackage{xcolor}

% Настройка listings для работы с UTF-8
\usepackage{listingsutf8}

\lstdefinestyle{bash}{
  language=bash,
  basicstyle=\ttfamily\small,
  breaklines=true,
  columns=fullflexible,
  frame=single,
  framerule=0.3pt,
  xleftmargin=0.5em,
  xrightmargin=0.5em,
  showstringspaces=false,
  inputencoding=utf8  % Важно для русского текста
}

\lstset{
  style=bash,
  literate={а}{{\selectfont\char224}}1
           {б}{{\selectfont\char225}}1
           {в}{{\selectfont\char226}}1
           {г}{{\selectfont\char227}}1
           {д}{{\selectfont\char228}}1
           {е}{{\selectfont\char229}}1
           {ё}{{\"e}}1
           {ж}{{\selectfont\char230}}1
           {з}{{\selectfont\char231}}1
           {и}{{\selectfont\char232}}1
           {й}{{\selectfont\char233}}1
           {к}{{\selectfont\char234}}1
           {л}{{\selectfont\char235}}1
           {м}{{\selectfont\char236}}1
           {н}{{\selectfont\char237}}1
           {о}{{\selectfont\char238}}1
           {п}{{\selectfont\char239}}1
           {р}{{\selectfont\char240}}1
           {с}{{\selectfont\char241}}1
           {т}{{\selectfont\char242}}1
           {у}{{\selectfont\char243}}1
           {ф}{{\selectfont\char244}}1
           {х}{{\selectfont\char245}}1
           {ц}{{\selectfont\char246}}1
           {ч}{{\selectfont\char247}}1
           {ш}{{\selectfont\char248}}1
           {щ}{{\selectfont\char249}}1
           {ъ}{{\selectfont\char250}}1
           {ы}{{\selectfont\char251}}1
           {ь}{{\selectfont\char252}}1
           {э}{{\selectfont\char253}}1
           {ю}{{\selectfont\char254}}1
           {я}{{\selectfont\char255}}1
           {А}{{\selectfont\char192}}1
           {Б}{{\selectfont\char193}}1
           {В}{{\selectfont\char194}}1
           {Г}{{\selectfont\char195}}1
           {Д}{{\selectfont\char196}}1
           {Е}{{\selectfont\char197}}1
           {Ё}{{\"E}}1
           {Ж}{{\selectfont\char198}}1
           {З}{{\selectfont\char199}}1
           {И}{{\selectfont\char200}}1
           {Й}{{\selectfont\char201}}1
           {К}{{\selectfont\char202}}1
           {Л}{{\selectfont\char203}}1
           {М}{{\selectfont\char204}}1
           {Н}{{\selectfont\char205}}1
           {О}{{\selectfont\char206}}1
           {П}{{\selectfont\char207}}1
           {Р}{{\selectfont\char208}}1
           {С}{{\selectfont\char209}}1
           {Т}{{\selectfont\char210}}1
           {У}{{\selectfont\char211}}1
           {Ф}{{\selectfont\char212}}1
           {Х}{{\selectfont\char213}}1
           {Ц}{{\selectfont\char214}}1
           {Ч}{{\selectfont\char215}}1
           {Ш}{{\selectfont\char216}}1
           {Щ}{{\selectfont\char217}}1
           {Ъ}{{\selectfont\char218}}1
           {Ы}{{\selectfont\char219}}1
           {Ь}{{\selectfont\char220}}1
           {Э}{{\selectfont\char221}}1
           {Ю}{{\selectfont\char222}}1
           {Я}{{\selectfont\char223}}1
}

\setlength{\parindent}{1.25cm}
\setlength{\parskip}{0.15em}

\begin{document}
\begin{titlepage}
\centering
{\large МОСКОВСКИЙ АВИАЦИОННЫЙ ИНСТИТУТ \par}
{\large (НАЦИОНАЛЬНЫЙ ИССЛЕДОВАТЕЛЬСКИЙ УНИВЕРСИТЕТ) \par}
\vspace{1.6cm}
{\large Институт №8 «Компьютерные науки и прикладная математика» \par}

\vfill

{\LARGE \textbf{Отчет по лабораторным работам} \par}
\vspace{0.4cm}
{\Large по курсу «Информационный поиск» \par}

\vfill

\begin{flushright}
Выполнил: Бачурин Павел Дмитриевич\\
Группа: М8О-403Б-22\\
Преподаватель: Кухтичев Антон Алексеевич
\end{flushright}

\vfill

{Москва, 2025}
\end{titlepage}

\section*{Введение}

Цель работы — разработать минимальный поисковый движок по собственному корпусу документов. Корпус собирается поисковым роботом (crawler) и хранится в MongoDB. Поиск реализован на C++ и включает токенизацию, стемминг, построение частотного распределения и проверку закона Ципфа, а также булев индекс и булев поиск. Для взаимодействия предусмотрены два интерфейса: CLI и веб-сервис (HTML-форма).

\section*{Анализ корпуса документов}

\subsection*{1. Описание корпуса и характеристик документов}

В качестве источников использованы русскоязычные новостные порталы Match TV и Tass. Документы представляют собой HTML-страницы новостных статей спортивной и общественно-политической тематики.

\begin{itemize}
\item Match TV: спортивные новости, трансляции, интервью, аналитические материалы.
\item Tass: новости политики, экономики, спорта, культуры, науки и технологий.
\end{itemize}

Тип документа в корпусе: веб-страница новости/статьи в формате HTML

Язык: русский (основной контент), встречаются англицизмы, имена собственные и термины на английском языке

\subsubsection*{1.1. Из чего состоит «сырой» документ (HTML)}

У обоих источников «сырой» документ (HTML) обычно содержит:

\begin{itemize}
\item основной контент статьи
\item заголовок (H1) и подзаголовки (H2-H4)
\item дата/время публикации (иногда с указанием времени обновления)
\item рубрика/раздел/теги
\item текст статьи (абзацы с форматированием: жирный, курсив, цитаты)
\item мультимедийный контент: изображения с подписями, видео-вставки, инфографика
\item ссылки на связанные материалы (по теме, архивные публикации)
\item навигационные элементы: хлебные крошки, меню категорий
\item блоки социальных кнопок (поделиться в соцсетях)
\item элементы авторизации/регистрации, подписки на рассылку
\item комментарии пользователей (на Match TV, у Tass обычно без комментариев)
\item мета-теги для SEO (description, keywords, Open Graph)
\item канонический URL (canonical link)
\end{itemize}

\subsection*{2. Выделение текста из «сырых» HTML документов}

\subsubsection*{2.1. Что считать «текстом документа»}

При индексации из HTML выделяется только содержательный текст:

\begin{itemize}
\item заголовок статьи (обычно в теге h1 или в специальном контейнере);
\item основной текст статьи (абзацы в тегах p, div с классом статьи);
\item лид/аннотация (краткое введение перед текстом);
\item подписи к изображениям и видео;
\item имена авторов и источников информации (при наличии).
\end{itemize}

В проекте используется упрощённое «снятие» тегов: удаление HTML-тегов и специальных секций (script/style/noscript) с последующей нормализацией пробелов. Для Match TV и Tass особенно важно корректно выделять основной контент, так как на новостных сайтах много навигационных и рекламных блоков. Для повышения качества можно использовать DOM-парсер с CSS-селекторами, нацеленными на контейнеры с классом статьи (например, article-content, post-text, news-text).

\subsection*{3. Проверка "пригодности корпуса": существующий поиск по документам}

Требование ЛР: если нельзя искать по документам существующими поисковиками — корпус использовать нельзя.

\subsubsection*{3.1. Встроенный поиск по сайту}

\begin{itemize}
\item Match TV: присутствует страница поиска \url{https://matchtv.ru/search/}
\item Tass: присутствует страница поиска \url{https://tass.ru/search}
\end{itemize}

Вывод: корпус можно использовать — существует встроенный поиск у обоих источников.

\subsubsection*{3.2. Внешний поиск (Google / Яндекс) с ограничением на сайт}

Можно искать по домену с оператором site: в Google:
\begin{itemize}
\item site:matchtv.ru [запрос]
\item site:tass.ru [запрос]
\end{itemize}

Также у Яндекса есть аналогичные операторы для ограничения поиска по сайту.

\subsection*{4. Примеры запросов к существующим поисковикам и недостатки выдачи}

\subsubsection*{4.1. Примеры запросов (встроенный поиск)}

Match TV - запрос: "футбол чемпионат мира"

Tass - запрос: "экономика санкции" с фильтрацией по дате

\subsubsection*{4.2. Примеры запросов (Google/Яндекс с site:)}

Google: site:matchtv.ru "хоккей КХЛ"

Google: site:tass.ru "выборы президента"

Яндекс: site:tass.ru "космическая программа" + последние новости

\subsubsection*{4.3. Недостатки поисковой выдачи}

\begin{itemize}
\item Дублирование новостей: одно и то же событие может быть опубликовано в нескольких новостных лентах с разными заголовками и акцентами.
\item Временная релевантность: свежие новости часто ранжируются выше исторически важных, даже если последние более релевантны запросу.
\item Проблемы с поиском архивных материалов: старые статьи могут быть плохо проиндексированы или недоступны через встроенный поиск.
\item Зависимость от формулировки: спортивные термины и политические формулировки часто требуют точного совпадения, синонимы могут давать разные результаты.
\end{itemize}

\subsection*{5. Статистика по корпусу}

\subsubsection*{5.1. Общая статистика корпуса}

\begin{table}[ht]
\centering
\small
\begin{tabularx}{\textwidth}{@{}Xrrr@{}}
\toprule
Показатель & matchtv.ru & tass.ru & Итого \\
\midrule
Кол-во документов & 25000 & 35000 & 60000 \\
Raw объём, MB & 415.27 & 521.43 & 936.70 \\
Выделенный текст, символов & 185 432 150 & 378 916 278 & 564 348 428 \\
Средний текст, символов/док & 7417.29 & 10826.18 & 9405.81 \\
Среднее кол-во слов/док & 1236.22 & 1804.36 & 1567.64 \\
\bottomrule
\end{tabularx}
\end{table}

\subsubsection*{5.2. Особенности статистики}

\begin{itemize}
\item Tass имеет более длинные статьи в среднем (общественно-политические, аналитические материалы)
\item Match TV статьи короче, но чаще обновляются (спортивные новости, трансляции)
\item Оба источника содержат структурированные даты публикации, что полезно для временных фильтров
\item Высокая плотность именованных сущностей: имена спортсменов, команд, политиков, организаций
\end{itemize}

\subsection*{6. Итоговый вывод}

Корпус из материалов matchtv.ru и tass.ru пригоден для выполнения последующих лабораторных работ, т.к.:

\begin{itemize}
\item документы имеют четкую структуру новостной статьи с выделяемым основным текстом
\item существует проверяемый поиск по исходным документам: встроенный поиск обоих сайтов и внешний поиск с ограничением site
\item разнообразие тематик (спорт и общественно-политические новости) позволяет тестировать поиск на разных типах запросов
\item наличие мета-данных (дата, категория, теги) предоставляет дополнительные возможности для фильтрации и ранжирования
\item корпус представляет актуальный языковой материал современного русского языка в новостной сфере
\end{itemize}

\section*{Поисковая система и Crawler}

\subsection*{1. Архитектура поисковой системы}

\subsubsection*{1.1. Общая схема работы}

Поисковый робот обходит страницы целевых сайтов, извлекает ссылки на документы, скачивает HTML и сохраняет результат в MongoDB. Робота можно остановить и запустить снова — он продолжает обход с места остановки, используя коллекцию frontier.

\subsubsection*{1.2. Формат хранимого документа}

В коллекции документов сохраняются поля:

\begin{itemize}
\item content\_hash — хеш содержимого документа;
\item crawl\_date — дата обкачки (Unix time stamp);
\item html\_content — «сырой» HTML документа;
\item source\_name — название источника (matchtv или tass);
\item url — нормализованный URL документа;
\end{itemize}

\subsubsection*{1.3. Повторная обкачка и проверка изменений}

При повторной обкачке документ обновляется только в случае изменений. Проверка может выполняться по хешу HTML, заголовкам ETag/Last-Modified или по сравнению нормализованного текста.

\subsubsection*{1.4. Конфигурация (YAML)}

Робот получает единственный аргумент — путь к YAML-конфигу.

\subsection*{2. Индексация и поисковые компоненты (C++)}

\subsubsection*{2.1. Источник данных для индексации}

Индексатор считывает из stdin документы в JSON формате, содержащие поля html\_content, url, source\_name. В стандартный ввод данные попадают из MongoDB при помощи утилиты mongoexport.

\subsubsection*{2.2. Токенизация}

Токенизация выполняется по следующим правилам:

\begin{itemize}
\item текст переводится в нижний регистр;
\item токеном считается последовательность букв/цифр (включая кириллицу);
\item знаки пунктуации и служебные символы выступают разделителями.
\end{itemize}

Недостатки: возможны «неудачные» токены (например, c++17, e-mail, url, 3.14, слова с дефисами).

Улучшение: отдельные правила для дефисов, апострофов, сокращений, а также выделение токенов из camelCase.

\subsubsection*{2.3. Статистика токенизации и время работы}

Индексатор выводит требуемые статистики: количество токенов, среднюю длину токена, а также время выполнения и скорость токенизации (KB/s).

Пример запуска в Docker:

\begin{lstlisting}
mongoexport \
    --db $MONGO_DB \
    --collection $MONGO_COLLECTION \
    --username $MONGO_USERNAME \
    --password $MONGO_PASSWORD \
    --authenticationDatabase admin \
    --fields _html_content,url,source_name \
    --quiet | ./indexer
\end{lstlisting}

Вывод программы (полученные значения):

\begin{lstlisting}
Documents: 35000
Unique terms: 805701
Total tokens: 9240789
Avg token length: 16.3485
Input size: 2005990.1 KB
Time: 2689.8331 sec
Speed: 745.99 KB/sec
\end{lstlisting}

\subsubsection*{2.4. Закон Ципфа}

После индексации строится распределение частот терминов по рангам.

Индексатор сохраняет CSV-файл (zipf.csv), содержащий rank, freq и значение аппроксимации Zipf C/r.

\begin{figure}[ht]
  \centering
  \includegraphics[width=0.8\linewidth]{images/zipf.png}
  \caption{График распределения Ципфа}
  \label{fig:zipf}
\end{figure}

\subsubsection*{2.5. Стемминг}

Для нормализации словоформ используется упрощённый стеммер (поддержка русского и английского). Нормализация применяется на этапе индексации и на этапе обработки запроса.

Оценка качества: сравнение выдачи до/после стемминга на нескольких запросах, включая случаи ухудшения (омонимия, чрезмерное обрезание суффиксов).

\subsubsection*{2.6. Булев индекс}

Булев (инвертированный) индекс хранит для каждого термина список идентификаторов документов, в которых он встречается. Для словаря и списков документов используются собственные структуры данных (без map/unordered\_map).

\subsubsection*{2.7. Булев поиск и интерфейсы}

Поддерживаются операции AND, OR, NOT и круглые скобки. Результат поиска — список документов (url и источник). Реализованы два интерфейса:

\begin{itemize}
\item CLI: запросы читаются из аргумента запуска или stdin, результаты выводятся в stdout.
\item Web: HTTP-сервер с HTML-формой ввода и HTML-выдачей.
\end{itemize}

\subsection*{3. Примеры запросов и проверка работы}

\subsubsection*{3.1. CLI}

Пример запуска CLI-поиска (индекс должен быть построен заранее):

\begin{lstlisting}
./engine 'search query with && operator'
\end{lstlisting}

Пример вывода:

\begin{lstlisting}
Found 3 documents:
- https://matchtv.ru/channel/matchtv/programs
- https://matchtv.ru/author/vasilii-bogdanov
- https://matchtv.ru/author/evgenii-dzichkovskii
\end{lstlisting}

\subsubsection*{3.2. Web}

Пример запуска веб-сервиса:

\begin{lstlisting}
python3 main.py
\end{lstlisting}

Пример вывода:

\begin{lstlisting}
Базовая директория: /home/user/MAI/InfoSearch/engine
Путь к engine: /home/user/MAI/InfoSearch/engine/engine
Путь к forward.idx: /home/user/MAI/InfoSearch/engine/forward.idx
Путь к inverted.idx: /home/user/MAI/InfoSearch/engine/inverted.idx
Сервер запущен: http://localhost:8000
\end{lstlisting}

Примеры запросов для проверки:

\begin{itemize}
\item футбол \&\& тайм
\item футбол || хоккей
\item матч \&\& спорт
\item (матч \&\& футбол) || хоккей
\end{itemize}

\begin{figure}[ht]
  \centering
  \includegraphics[width=0.8\linewidth]{images/ui_1.png}
  \caption{Веб-интерфейс поисковой системы}
  \label{fig:ui1}
\end{figure}

\begin{figure}[ht]
  \centering
  \includegraphics[width=0.8\linewidth]{images/ui_2.png}
  \caption{Результаты поиска}
  \label{fig:ui2}
\end{figure}

\begin{figure}[ht]
  \centering
  \includegraphics[width=0.8\linewidth]{images/ui_3.png}
  \caption{Просмотр документа}
  \label{fig:ui3}
\end{figure}

\section*{Заключение}

\subsection*{Итоговые выводы}

В ходе выполнения данных лабораторных работ я познакомился с тем, как устроены современные поисковые системы, и даже сумел реализовать свою собственную.

Моя поисковая система:
\begin{itemize}
\item содержит расширяемый бинарный индекс, включающий обратный и прямой индексы, пригодный для булева поиска.
\item реализует булев поиск с поддержкой AND, OR, NOT, пробелов и скобок;
\item запускается в Docker;
\item предоставляет два интерфейса – CLI и Web;
\item предоставляет возможность смотреть на raw HTML выдаваемых документов.
\end{itemize}

Скорость выполнения запросов находится в миллисекундном диапазоне для типовых случаев; длительная работа возникает на выражениях с большими объединениями и отрицаниями частых термов.

В целом поисковую систему можно дорабатывать и добавлять к ней новые функции, которые позволят улучшить качество и скорость поиска.

\end{document}
